%%
%% This is file `example/ch_concln.tex',
%% generated with the docstrip utility.
%%
%% The original source files were:
%%
%% install/buptgraduatethesis.dtx  (with options: `ch-concln')
%%
%% This file is a part of the example of BUPTGraduateThesis.
%%

\chapter{功能测试}

参考文献可以使用\cite{BUPT_Thesis_Format_2014}和\parencite{BUPT_Thesis_Format_2004}的表示方法。

新增自动缩略语功能,可以看到第一次使用 ``\gls{3GPP}'' 和随后的使用``\gls{3GPP}{''}会展现出不同的效果。
但需要注意的是使用缩略语时英文两侧的自动空格不会出现,比如这个\gls{3GPP}和这个3GPP。
但是,当英文在句尾的时候其实是不需要在标点前加空格的,所以在定义缩略语的时候前后都加上空格也是不可行的。
现在暂时只能要求大家手动调整一下 \texttt{\textbackslash{}gls\{\}} 两侧的空格了。

\section{三国演义}
《三国演义》\cite{SANGUOYANYI}是中国第一部长篇章回体历史演义的小说,以描写战争为主,反映了蜀(汉)、魏、吴三个政治集团之间的政治和军事斗争,大致分为黄巾之乱、董卓之乱、群雄逐鹿、三国鼎立、三国归晋五大部分。

在广阔的背景下,上演了一幕幕波澜起伏、气势磅礴的战争场面,成功刻画了近五百个人物形象,其中曹操、刘备、孙权、诸葛亮、周瑜、关羽、张飞等人物形象脍炙人口,其中诸葛亮是作者心目中的“贤相”的化身,他具有“鞠躬尽瘁,死而后已”的高风亮节,具有近世济民再造太平盛世的雄心壮志,而且作者还赋予他呼风唤雨、神机妙算的奇异本领。
曹操是一位奸雄,他生活的信条是“宁教我负天下人,休教天下人负我”,既有雄才大略,又残暴奸诈,是一个政治野心家阴谋家这与历史上的真曹操是不可混同的。
关羽“威猛刚毅”、“义重如山”。
但他的义气是以个人恩怨为前提的,并非国家民族之大义。
刘备被作者塑造成为仁民爱物、视贤下士、知人善任的仁君典型。

\subsection{长坂坡}
京剧《长坂坡》\cite{CHANGBANPO}是依据《三国演义》改编的京剧传统剧目。

故事叙述:刘备自烧屯新野之后,弃樊城,阻襄阳,一路率引军民,流离败走,穷促万分。
关羽、诸葛亮,已先后遣往夏口,乞救于刘琦未返,刘备等往投江陵暂驻,中途经过当阳,驻扎景山之下。
忽然曹操大兵,漫山遍野追至,夤夜厮杀,刘备众大败,及天明检点随从只余百余骑,刘备家眷及赵云、简雍、二糜等将,均不知下落,其余百姓,亦均散失殆尽。
此时赵云因于阿斗及甘、糜二夫人等失散,遂单骑冲突,四处找寻主眷,沓无下落。
往回三数次,遇见简雍被创卧地,始略知失踪处所。
赵云先救出简雍,令回,再往军中及百姓中搜访,先救甘夫人于难民队,同时又救糜竺,亲自护送至长坂坡,令糜竺保甘嫂先行,折身再回,觅糜嫂及阿斗。
途中刺落夏侯恩,收获青釭宝剑,七次冲入重围,方得百姓指引,得见糜夫人抱阿斗坐于坍墙枯井之旁啼哭。
夫人身受数创,不能行走。
赵云叩见,极力请夫人上马,欲保护而出。
夫人深知大义,惟以阿斗为托,己则以愿死报主,免累赵云,赵云再三安慰催行,力任无妨,夫人再三不可,亦促赵云速行。
继见赵云坚待不去,恐且迟延遇寇,乃跳身入井,以速赵云之行。
赵云大惊,尚踌躇设法营救,则曹军人马已至,不得已推墙掩井,解甲藏阿斗于胸前,忽忽上马,厮杀夺围欲出。
此时曹操大兵云集,群矢于赵云一身,赵云在核心,东斩西杀,虽不败辱,而屡濒于厄。
幸曹操爱勇将,赖徐庶乘间说曹操,以生擒勿伤,传令全军,始得完肤而返。

\subsection{测试图片}

\begin{figure}
    \begin{subfigure}[b]{.5\linewidth}
    \centering\large A
    \caption{A subfigure}\label{fig:1a}
    \end{subfigure}%
    \begin{subfigure}[b]{.5\linewidth}
    \centering\large B
    \caption{Another subfigure}\label{fig:1b}
    \end{subfigure}
    \caption[A figure]{A figure \protect\cite{CITATION_ARTICLE}}\label{fig:1}
\end{figure}

引用\cref{fig:1}和其中的\cref{fig:1a,fig:1b}。

\subsection{测试参考文献}
测试所有参考文献类型\cite{CITATION_BOOK,CITATION_ARTICLE,CITATION_PROCEEDINGS,CITATION_INPROCEEDINGS,CITATION_TECHREPORT,CITATION_STANDARD,CITATION_PATENT,CITATION_NEWSPAPER,CITATION_ELECTRONIC}。

\chapterbib
